\section{Conclusion} \label{conclusion}
\thispagestyle{plain} % surpress header on first page

I revisit the structural model pioneered in \cite{Rust.1987} by evaluating how sensitive the model predictions react to wrong specifications of the model and its numerical implementation during the calibration process as well as when obtaining the model predictions. I simulate the model according to a true specification of the mathematical and computational model from which I deviate deliberately when estimating the Rust model in a Monte Carlo simulation. I compare the calibration procedures of the Nested Fixed Point Algorithm and Mathematical Programming with Equilibrium Constraints in this setting and show how the concept of uncertainty propagation together with a Monte Carlo simulation lends itself well for this comparison. In this context, I compare the distribution of a counterfactual demand level as quantity of interest (as an indicator for the quality of prediction of the model) across different model and numerical specifications as well as the calibration approach (NFXP or MPEC).

I find that irrespective of the calibration approach, the misspecifications have an impact on the distribution of the QoI suggesting that a careful quantification of potential model and numerical errors should a routine task when calibrating a structural model. Surprisingly, I find that the fact that NFXP and MPEC are implemented differently can result in different distributions of the QoI when allowing for more flexibility in the cost function. This suggests that the two approaches should be part of a careful uncertainty quantification when calibrating a model.

While my approach is computationally expensive, it reveals the need for systematic sensitivity analysis of model predictions to different specifications. The UQ literature offers many structured approaches to tackle parameter uncertainty paired with the use of computationally lighter surrogate models as reported in \cite{Saltelli.2008}. A combination of those techniques together with changing specifications of the mathematical and computational model is an interesting field of further research in Economics which has been gaining some attention lately already (see \cite{Scheidegger.2019} and \cite{Harenberg.2019}).