\section{Mathematical Programming with Equilibrium Constraints}
\thispagestyle{plain} % surpress header on first page

Mathematical Programming with Equilibrium Constraints can be traced back notation and concept wise to Game Theory. Specifically in the theory on Stackelberg games MPEC found its first development. It was due to \cite{Luo.Pang.Ralph.1996}, though, that MPEC was set onto a mathematically rigorous foundation. They argue to have done so in order to present its manifold possibilities of application that had been overlooked previously. 

Before we go into the details of MPEC regarding its mathematical formulation, applications and use in Economics, let us break down the lengthy term into its key components. The beginning "Mathematical Program" solely captures that we look at a mathematical optimization problem. The particularity of this problem comes in with the "Equilibrium Constraints". Mathematically this means that this optimization problem is subject to variational inequalities (VI) as constraints. \cite{Nagurney.1993} explains that VIs consist of but are not limited to nonlinear equations, optimization as well as fixed point problems. More broadly spoken VIs are able to harnesses our intuitive notion of economic equilibrium for which typically a functional or a system of equations must be solved for all possible values of a given input. This is tightly linked to what is looked for when solving a Stackelberg game. Essentially, an economic equilibrium has to be found. As a reminder in a Stackelberg game, there is one leader that moves first followed by the moves of some followers. Solving this problem involves the leader to solve an optimization problem that in turn is subject to an optimization procedure of the followers given every possible optimal value the leader might find. The variational inequality here is the problem of the followers which involves solving a decision problem for every possible move of the leader and which is cast into the optimization problem of the leader as a constraint. It can be seen from the fact that the leader moves first and followers move after, as noted by \cite{Luo.Pang.Ralph.1996}, that the MPEC formulation is a hierarchical mathematical concept which captures multi-level optimization and hence can prove useful for the modeling of decision-making processes. They further explain that this feature can further be beneficial in other fields than just Economics. They showcase that a classification problem in machine learning can be formulated as an MPEC and they further describe some problems in robotics, chemical engineering and transportation networks in MPEC notation. 

While this discussion shows that MPEC problems appear in theoretical Economics, \cite{Su.Judd.2012} enter with the novel idea to formulate an estimation procedure in structural econometrics as a MPEC. In the following I present their idea using the notation they originally suggested. 

\paragraph{}
In order to estimate the structural parameters of an economic model using data, researchers commonly rely on the Generalized Method of Moments or maximum likelihood estimation. If the researchers opt for the most complex way of estimation (as opposed to using methods lowering the computational burden such as in \cite{Hotz.Miller.1993}) which involves solving the economic model at each guess of the structural parameters, they frequently employ the nested fixed point algorithm (NFXP) suggested by \cite{Rust.1987}. In the case of maximum likelihood estimation, the approach works like the following: An unconstrained optimization algorithm guesses the structural parameters and for each of those guesses the underlying economic model is solved. The resulting outcome of the economic model allows to evaluate the maximum likelihood which then gives new information to the optimization algorithm to form a new guess of the structural parameters. This is repeated until some stopping criteria is met. To make it more explicit, let us introduce some mathematical notation. Let us assume that an economic model is described by some structural parameter vector $\theta$ and a state vector $x$ as well as some endogenous vector $\sigma$. Assume we further observe some data consisting of $X = \{x_i, d_i\}^M_{i=1}$. Here, $x_i$ is the observed state and $d_i$ is the observed equilibrium outcome of the underlying economic decision model. $M$ is the number of data points.

Let us further assume that generally $\sigma$ depends on the parameters $\theta$ through a set of equilibrium conditions (or in the previous notation of variational inequalities), i.e. $\sigma(\theta)$. This includes e.g. Bellman equations. The consistency of $\sigma$ with $\theta$ is expressed by the following condition: 

\begin{equation*}
h(\theta, \sigma) = 0.
\end{equation*}

For a given $\theta$, let $\Sigma(\theta)$ denote the set of $\sigma(\theta)$ for which the equilibrium conditions hold, i.e. for which $h(\theta,\sigma)=0$. 

\begin{equation*}
\Sigma(\theta) := \{\sigma:h(\theta, \sigma)=0\}.
\end{equation*}

Let $\hat{\sigma}(\theta)$ denote an element of the above set. In the case of an infinite horizon dynamic discrete-choice model, this represents the expected value function evaluated at a specific parameter vector $\theta$. In the case that a unique fixed point for the expected value function exists, $\hat{\sigma}(\theta)$ would be a single value but this does not have to hold in general. If the equilibrium condition involves solving a game for instance, one could easily imagine to find multiple equilibria which causes $\Sigma(\theta)$ to have multiple elements for a given $\theta$. 

For the case of multiple $\hat{\sigma}(\theta)$ the solution to the maximization of the log likelihood function $L(.)$ given the data $X$ becomes:

\begin{equation}
\hat\theta = \argmax_{\theta} \{ \max_{\hat\sigma(\theta)\in\Sigma(\theta)} L(\theta,\hat\sigma(\theta); X)\}. \label{eq1}
\end{equation}

This shows that the above problem boils down to finding the parameter vector $\theta$ that gives out possibly several $\hat{\sigma}(\theta)$ and which yields in combination with one of them the highest possible log likelihood of all combinations of $\theta$ and $\hat{\sigma}(\theta)$.

As already shortly described, the NFXP attempts to solve this problem in a nested loop. First, a guess for $\hat{\theta}$ is fixed for which the corresponding $\hat{\sigma}(\hat\theta)$ (possibly multiple) are found. For those possibly multiple combinations of $\hat\theta$ and $\hat\sigma(\hat\theta)$ the one that yields the highest log likelihood is chosen and this procedure is repeated until the $\hat{\theta}$ is found that solves equation (\ref{eq1}). The NFXP therefore solves this problem by running an unconstrained optimization of the log likelihood function that involves solving the economic model at each parameter guess. For the simplified version of $\hat{\sigma}(\hat\theta)$ being single-valued this idea is captured in the following pseudocode:

\vspace{2ex}
\begin{algorithm}[H]
	\SetAlgoLined
	\KwIn{$\hat\theta_n$, $n=0$, $X$\;}
	\While{$f(|| \hat\theta_{n+1} - \hat\theta_{n} ||) \geq$ stopping tolerance}{
		Calculate $\hat{\sigma}(\hat\theta_n)$ and evaluate $L(\hat\theta_n,\hat\sigma(\hat\theta_n); X)\}$\;
		Based on that fix a new guess $\hat\theta_{n+1}$\;
	}
	\caption{Nested Fixed Point Algorithm}
\end{algorithm}
\vspace{2ex}
 
The above formulation clearly conveys two points already. The problem posed in equation (\ref{eq1}) is essentially a hierarchical one. Additionally, we work with equilibrium conditions. This gives an indication that an MPEC formulation of the above problem might exist. \cite{Su.Judd.2012} formally prove exactly this. The difference to the NFXP way of writing the problem is that one know ensures differently that a guess of $\theta$ is consistent with the equilibrium condition $h(\theta, \sigma)=0$. In the MPEC formulation $\sigma$ is modeled explicitly as another parameter vector that can be chosen freely by an optimization algorithm instead of being derived from $\theta$. This gives rise to a new log likelihood function $L(\theta, \sigma; X)$ for which they coin the term $\textit{augmented likelihood function}$. Still they have to make sure, though, that the equilibrium condition holds meaning that the parameter guess for $\theta$ is consistent with the equilibrium $\sigma$. This is done by imposing it as a constraint to the augmented log likelihood function. The optimization problem now becomes a constrained optimization looking like the following:

\begin{equation}
	\begin{aligned}
		& \max_{(\theta, \sigma)} L(\theta, \sigma; X) \\
		& \text{subject to } h(\theta, \sigma) = 0.
	\end{aligned}
	\label{eq2}
\end{equation}

\cite{Su.Judd.2012} provide a proof that the two formulations in the equations (\ref{eq1}) and (\ref{eq2}) are actually equivalent in the sense that they yield the same solution $\hat\theta$ for the structural parameters of the model. The general setup of the algorithm used for MPEC simplifies to the following:

\vspace{2ex}
\begin{algorithm}[H]
	\SetAlgoLined
	\KwIn{$\hat\theta_n$, $\hat{\sigma}_n$, $n=0$, $X$\;}
	\While{$f(|| (\hat\theta_{n+1}, \hat{\sigma}_{n+1}) - (\hat\theta_{n}, \hat{\sigma}_{n}) ||) \geq$ stopping tolerance}{
		Evaluate $L(\hat\theta_n, \hat\sigma_n; X)$\;
		Based on that fix a new guess $(\hat\theta_{n+1}, \hat{\sigma}_{n+1})$\;
	}
	\caption{Mathematical Programming with Equilibrium Constraints}
\end{algorithm}
\vspace{2ex}

Having established that the two algorithms or formulations theoretically yield the same solution for the structural parameters, \cite{Dong.Hsieh.Zhang.2017} note that the different way they achieve that can be characterized in the following way: The NFXP solves the problem with an unconstrained optimization algorithm by posing the problem as a low dimensional one. The MPEC formulation on the other hand is a high dimensional problem that needs to be solved using an optimizer that can handle constrained optimization problems involving nonlinear constraints. The difference in dimensionality stems from the fact that in the MPEC also the equilibrium variables need to be chosen. This observation automatically raises the question whether there is any advantage MPEC might have over NFXP as at first sight the problem seems to be more complicated. \cite{Su.Judd.2012} identify one major advantage which rests on the fact that the solving of the economic model does not have to be taken care of by the researcher but is cast to the optimization algorithm. The first immediate advantage comes from less coding effort. In the case of the problem of infinite-horizon dynamic discrete choice posed in \cite{Rust.1987} which \citeauthor{Su.Judd.2012} base their comparison on, this makes a significant difference. Another advantage comes from the way modern solvers such as KNITRO (based on \cite{Byrd.Nocedal.Waltz.2006}) or IPOPT (see \cite{Pirnay.Lopez-Negrete.Biegler.2011}) handle constraints. Those constraints are not solved exactly until the last guess of the structural parameters which allows them to potentially perform faster than the NFXP in which at each guess of the structural parameters $\theta$ is calculated with high precision. This can especially be a factor when the underlying model is quite complicated such as for instance a game with multiple equilibria. \cite{Dube.Fox.Su.2012}, who look at the NFXP and MPEC for a BLP demand model, see another more practical factor that might make the case for MPEC. They report that practioners tend to loosen the convergence tolerance for solving the economic model when using the NFXP in order to speed up the process (especially when the model is computationally intensive). This leads to an increasing numerical error in the equilibrium outcome which might propagate into the guess of the structural parameters as the equilibrium outcome influences the likelihood function. This can result in wrong parameter estimates or even in failure of convergence. They further report that the existing literature on durable and semi-durable good markets might profit from MPEC as there are models that would need three nested loops when using the NFXP but two of them could be easily cast into the constraints of one major loop when opting for MPEC. 

MPEC has one key limitation, though, that is mentioned by several different authors. \cite{Wright.2004} reports that the speed of modern solvers based on interior point algorithms (such as the before mentioned KNITRO and IPOPT) crucially depends on the sparsity of the Jacobian and the Hessian of the Langrangian. This highlights that the higher dimensionality (the size of the Jacobian and the Hessian) of MPEC problems does not need to generally cause a problem but if it comes with few zero elements in the before mentioned matrices it might, i.e. when those matrices are rather dense. This, in turn, depends on the economic model at hand and hence gives an indication that whether NFXP or MPEC should be prefered might depend on the specific context. This is confirmed by \citeauthor{Dube.Fox.Su.2012} who find that MPEC is faster and more reliable (looking at the convergence rate) than the NFXP but for problems that cause the constraint Jacobian and Hessian to be sparse but this advantage deteriorates when having dense matrices. \cite{Jorgensen.2013} confirms this for the case of estimating a continuous choice model. He states that MPEC needs too much memory when the state space is large and the before mentioned matrices are dense. In a more recent study \cite{Dong.Hsieh.Zhang.2017} compare MPEC and NFXP for an empirical matching model. They obtain a more fine-grained image of the previously noted tradeoff. In their estimation, they solve the same model with a more sparse and a more dense version of MPEC. They obtain this difference by first setting up MPEC with all equilibrium conditions as constraints (sparse version) and then again with a version where they substitute in some of the equilibrium conditions into the other ones (dense version). For the comparison of the two, they find that the sparse version has better convergence rates while the dense version has a speed advantage. The authors observe another interesting element when comparing NFXP and MPEC. In their application the inner loop can potentially fail depending on the structural parameter guess provided. This adds another problematic element to the use of the NFXP. This is due to the set up of separating the structural guess from the solving of the model. The optimizer cannot take into account whether a structural parameter guess might cause the inner loop (the solving of the economic model) to fail. This is different for the MPEC formulation in which the algorithm can jointly consider the structural parameter and the equilibrium outcome.



